\documentclass[a4paper, 10.5pt]{article}
\usepackage{fullpage} % changes the margin
\usepackage[T1]{fontenc}
\usepackage[utf8]{inputenc}
\usepackage{dcolumn}
\usepackage{amsfonts}
\usepackage{graphicx}
\usepackage{amssymb}
\usepackage{amsmath}
\usepackage{hyperref}
\usepackage{float}
\usepackage[czech]{babel}

\usepackage{booktabs}
\newcommand{\ra}[1]{\renewcommand{\arraystretch}{#1}}

\begin{document}
% Header
\noindent
\large\textsc{\textbf{Jana Novotná \& Vladan Glončák}}  \\
\normalsize \textsc{Statistical NLP II}\\
\textsc{Assignemnt 3}

\noindent\rule[0.5ex]{\linewidth}{1pt}

\newcommand{\norm}[1]{\left\lVert#1\right\rVert}

\vskip 30 pt

Součástí úkolu jsou Python3 skripty \textbf{assignment3\_1.py} a \textbf{assignment3\_2.py}.

\vskip 20 pt

\section*{Task 1}

Použili sme Brillov tagger z knižnice NLTK.
Ako počiatočný tagger sme použili jednoduchý unigramový tagger, ktorý si  zapamätá najčastejšie značku pre každé slovo trénovacích dat a pre neznáme slová vráti špecialnú značku ``None''.
Použili sme počiatočnú množinu 24 vzorov\footnote{NLTK má tieto vzory ako súčasť knižnice} (template), ktoré pôvodne použil Brill. Pre angličtinu sme početpravidiel obmedzili\footnote{Limit sa pri týchto počtoch vždy naplní, čiže limit je rovnaký ako počet pravidiel.} na 247, pre češtinu na 500.
Tieto konkrétne hodnoty sme dostali skúšaním niekoľkých hodnôt.
Všeobecne platí, že väčšie množstvo pravidiel môže mierne zlepšiť presnosť, ale výrazne spomaňuje.

\subsection*{Výsledky}

\vskip 10pt

\begin{table}[H]
\centering
\begin{tabular}{ll}
\toprule
\textbf{Angličtina} & \textbf{Přesnost} \\
\midrule
1. stupeň & 0.8909342163585638 \\
2. stupeň & 0.8890974014126724 \\
3. stupeň & 0.8747591522157996 \\
4. stupeň & 0.8935028395769291 \\
5. stupeň & 0.8925100950892275 \\
\midrule
Průměr   & 0.888160740931 \\
$\sigma$ & 0.00686479392893 \\
\midrule
Průmer triv. & 0.865374314735\\
$\sigma$ triv. & 0.0143254860843 \\
\bottomrule
\end{tabular}
\hskip 50pt
\begin{tabular}{ll}
\toprule
\textbf {Čeština} & \textbf{Přesnost} \\
\midrule
1. stupeň & 0.7743503905627291 \\
2. stupeň & 0.7887507306445614 \\
3. stupeň & 0.7618758256274769 \\
4. stupeň & 0.7919649224755012 \\
5. stupeň & 0.7880596223212392 \\
\midrule
Průměr   & 0.781000298326 \\
$\sigma$ & 0.0113145302014 \\
\midrule
Průmer triv. & 0.738888267001\\
$\sigma$ triv. & 0.0134321134672 \\
\bottomrule
\end{tabular}
\caption{Přesnost (accuracy) pro Brillův tagger pro oba jazyky.}
\label{eng_big}
\end{table}

\section*{Task 2}

\subsection*{Supervizovaný HMM}

\begin{table}[H]
\centering
\begin{tabular}{lll}
\toprule
\textbf{Metóda} & \textbf{Čas} & \textbf{Přesnost} \\
\midrule
Žiadny pruning, add $\lambda$ & 7h 24m 22s  & 0.9252470471671056\\
Nevyhladený OOV model & 1m 34s & 0.9268375355252523\\
Vyhladený OOV model & 5h 35m 13s  & 0.9280369201887727\\
OOV a obmedzené stavy & 1h 12m 24s & 0.927802257971997\\
Add $\lambda$ a obmedzené stavy & 1h 23m 32s & 0.9250123849503298\\
OOV model, prunning 10 & 1m 53s & 0.9279847730294892\\
OOV model, prunning 20 & 3m 45s & 0.9280369201887727\\
\bottomrule
\end{tabular}
\caption{Přesnost (accuracy) pro různé metody pro anglčtinu.}
\label{eng_big}
\end{table}

\begin{table}[H]
\centering
\begin{tabular}{lrll}
\toprule
\textbf{Metóda} & \textbf{Pruning} & \textbf{Čas} & \textbf{Přesnost} \\
\midrule
Všetky stavy, add $\lambda$ & 2 & 19m17.624s & 0.7809926138477071\\
& 10 & 5h06m57s & 0.793984802593124\\
& 20 & 13h57m08s & 0.7988734789308677\\
\midrule
Obmedzené stavy, add $\lambda$ &2 & 3m22.020s &0.7676018917051916\\
&10& 13m48.912s& 0.7856953079334715\\
&20& 40m57.140s& 0.791274775492853\\
&30& 1h04m21s &0.7928423401881077\\
\midrule
Obmedzené stavy, OOV &2& 2m44.652s &0.7425208565811149\\
&10& 13m12.636s &0.7610393750996334\\
&20& 33m44.748s &0.7655029491471386\\
\bottomrule
\end{tabular}
\caption{Přesnost (accuracy) pro různé metody pro češtinu.}
\label{eng_big}
\end{table}

\begin{table}[H]
\centering
\begin{tabular}{ll}
\toprule
\textbf{Angličtina} & \textbf{Přesnost} \\
\midrule
1. stupeň & 0.9280369201887727 \\
2. stupeň & 0.9227200458727552 \\
3. stupeň & 0.9277601208774032 \\
4. stupeň & 0.9170442118418997 \\
5. stupeň & 0.93424514784421 \\
\midrule
Průměr   & 0.92596128932500821 \\
$\sigma$ & 0.00576517615271536 \\
\bottomrule
\end{tabular}
\hskip 50pt
\begin{tabular}{ll}
\toprule
\textbf{Čeština} & \textbf{Přesnost} \\
\midrule
1. stupeň & 0.791274775492853 \\
2. stupeň & 0.8096870184388119 \\
3. stupeň & 0.7636195508586526 \\
4. stupeň & 0.8034285110541747 \\
5. stupeň & 0.8035752174835561 \\
\midrule
Průměr   & 0.80233748927646664 \\
$\sigma$ & 0.0060165626151515 \\
\bottomrule
\end{tabular}
\caption{Přesnost pro oba jazyky určené cross-validací pro model z prořezávacím limitem 20. Pro češtinu se omezejume pouze na stavy, které jsou v trénovacích datech, pro angličtinu používáme jako stavy všechny dvojice značek.}
\end{table}

\subsection*{Nesupervizovaný HMM --- Baum-Welch}

Pre túto časť sme sa rozhodli použiť ako stavy iba bigramy, ktoré sme videli v trénovacích dátach, pretože inak by bol počet stavov bol veľký.

Navyše sme sa rozhodli použiť pruningovú metódu, aby sme tréning urýchlili.
% Vysvetlenie metódy

\end{document}
