\documentclass[a4paper, 10.5pt]{article}
\usepackage{fullpage} % changes the margin
\usepackage[T1]{fontenc}
\usepackage[utf8]{inputenc}
\usepackage{dcolumn}
\usepackage{amsfonts}
\usepackage{graphicx}
\usepackage{amssymb}
\usepackage{amsmath}
\usepackage{hyperref}
\usepackage{float}
\usepackage[czech]{babel}

\usepackage{booktabs}
\newcommand{\ra}[1]{\renewcommand{\arraystretch}{#1}}

\begin{document}
% Header
\noindent
\large\textsc{\textbf{Jana Novotná \& Vladan Glončák}}  \\
\normalsize \textsc{Statistical NLP II}\\
\textsc{Assignemnt 3}

\noindent\rule[0.5ex]{\linewidth}{1pt}

\newcommand{\norm}[1]{\left\lVert#1\right\rVert}

\vskip 30 pt

Součástí úkolu jsou Python3 skripty \textbf{assignment3\_1.py} a \textbf{assignment3\_2.py}.

\vskip 20 pt

\section*{Task 1}

Použili sme Brillov tagger z knižnice NLTK.
Ako počiatočný tagger sme použili jednoduchý unigramový tagger, ktorý si  zapamätá najčastejšie značku pre každé slovo trénovacích dat.
Použili sme počiatočnú množinu 18 pravidel, rovnaké ako v ukázke v knihe\footnote{Volne dostupné na \url{http://www.nltk.org/book/}}. Počet pravidiel je maximálne 200 (prednastavená hodnota NLTK).

\subsection*{Výsledky}

\vskip 10pt

\begin{table}[H]
\centering
\begin{tabular}{ll}
\toprule
\textbf{Angličtina} & \textbf{Přesnost} \\
\midrule
1. stupeň & 0.8878314603811958 \\
2. stupeň & 0.8877681340735528 \\
3. stupeň & 0.8591886684372233 \\
4. stupeň & 0.8916271557338613 \\
5. stupeň & 0.8927185098345708 \\
\midrule
Průměr   & 0.88382678569208084 \\
$\sigma$ & 0.01247811683956135 \\
\bottomrule
\end{tabular}
\hskip 50pt
\begin{tabular}{ll}
\toprule
\textbf {Čeština} & \textbf{Přesnost} \\
\midrule
1. stupeň & 0.7611722195653329 \\
2. stupeň & 0.7784951378925554 \\
3. stupeň & 0.7446763540290621 \\
4. stupeň & 0.7794976095512296 \\
5. stupeň & 0.7772650116698494 \\
\midrule
Průměr   & 0.76822126654160594 \\
$\sigma$ & 0.01355396354317235 \\
\bottomrule
\end{tabular}
\caption{Přesnost (accuracy) pro Brillův tagger pro oba jazyky.}
\label{eng_big}
\end{table}

\section*{Task 2}

\subsection*{Nesupervizovaný HMM --- Baum-Welch}

Pre túto časť sme sa rozhodli použiť iba bigramové a unigramové štatistiky, pretože ak by sme nechceli použiť použiť nuly ``natvrdo'' zabralo by to príšerne veľa času pre angličtinu, pre češtinu by to bolo jednoducho nerealizovateľné, časovo ani pamäťovo.

\end{document}
