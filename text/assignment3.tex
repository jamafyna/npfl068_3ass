\documentclass[a4paper, 10.5pt]{article}
\usepackage{fullpage} % changes the margin
\usepackage[T1]{fontenc}
\usepackage[utf8]{inputenc}
\usepackage{dcolumn}
\usepackage{amsfonts}
\usepackage{graphicx}
\usepackage{amssymb}
\usepackage{amsmath}
\usepackage{hyperref}
\usepackage{float}
\usepackage[czech]{babel}

\usepackage{booktabs}
\newcommand{\ra}[1]{\renewcommand{\arraystretch}{#1}}

\begin{document}
% Header
\noindent
\large\textsc{\textbf{Jana Novotná \& Vladan Glončák}}  \\
\normalsize \textsc{Statistical NLP II}\\
\textsc{Assignemnt 3}

\noindent\rule[0.5ex]{\linewidth}{1pt}

\newcommand{\norm}[1]{\left\lVert#1\right\rVert}

\vskip 30 pt

Součástí úkolu jsou Python3 skripty \textbf{assignment3\_1.py} a \textbf{assignment3\_2.py}.

\vskip 20 pt

\section*{Task 1}

Použili jsme Brillův tagger z knihovny NLTK.
Jako počátečný tagger jsme použili jednoduchý unigramový tagger, který si pouze zapamatuje nejčastejší značku pro každé slovo trénovacích dat.
Použili jsme počáteční množinu 18 pravidel, stejné jako v ukázce v knize\footnote{Volně dostupné na \url{http://www.nltk.org/book/}}. Počet pravidel je maximálne 150 (přednastavená hodnota NLTK).

\subsection*{Výsledky}

\vskip 10pt

\begin{table}[H]
\centering
\begin{tabular}{ll}
\toprule
\textbf{Angličtina} & \textbf{Přesnost} \\
\midrule
1. stupeň & 0.893425 \\
2. stupeň & 0.8929   \\
3. stupeň & 0.865525 \\
4. stupeň & 0.89685  \\
5. stupeň & 0.89665  \\
\bottomrule
\end{tabular}
\hskip 50pt
\begin{tabular}{ll}
\toprule
\textbf {Čeština} & \textbf{Přesnost} \\
\midrule
1. stupeň & 0.7756   \\
2. stupeň & 0.791825 \\
3. stupeň & 0.758075 \\
4. stupeň & 0.7922   \\
5. stupeň & 0.791    \\
\bottomrule
\end{tabular}
\caption{Přesnost (accuracy) pro Brillův tagger pro oba jazyky.}
\label{eng_big}
\end{table}

\section*{Task 2}


\end{document}
