\documentclass[a4paper, 10.5pt]{article}
\usepackage{fullpage} % changes the margin
\usepackage[T1]{fontenc}
\usepackage[utf8]{inputenc}
\usepackage{dcolumn}
\usepackage{amsfonts}
\usepackage{graphicx}
\usepackage{amssymb}
\usepackage{amsmath}
\usepackage{hyperref}
\usepackage{float}
\usepackage[czech]{babel}

\usepackage{booktabs}
\newcommand{\ra}[1]{\renewcommand{\arraystretch}{#1}}

\begin{document}
% Header
\noindent
\large\textsc{\textbf{Jana Novotná \& Vladan Glončák}}  \\
\normalsize \textsc{Statistical NLP II}\\
\textsc{Assignemnt 3}

\noindent\rule[0.5ex]{\linewidth}{1pt}

\newcommand{\norm}[1]{\left\lVert#1\right\rVert}

\vskip 30 pt

Součástí úkolu jsou Python3 skripty \textbf{assignment3\_1.py} a \textbf{assignment3\_2.py}.

\vskip 20 pt

\section*{Task 1}

Použili sme Brillov tagger z knižnice NLTK.
Ako počiatočný tagger sme použili jednoduchý unigramový tagger, ktorý si  zapamätá najčastejšie značku pre každé slovo trénovacích dat a pre neznáme slová vráti špecialnú značku ``None''.
Použili sme počiatočnú množinu 24 vzorov\footnote{NLTK má tieto vzory ako súčasť knižnice} (template), ktoré pôvodne použil Brill. Pre angličtinu sme početpravidiel obmedzili\footnote{Limit sa pri týchto počtoch vždy naplní, čiže limit je rovnaký ako počet pravidiel.} na 247, pre češtinu na 500.
Tieto konkrétne hodnoty sme dostali skúšaním niekoľkých hodnôt.
Všeobecne platí, že väčšie množstvo pravidiel môže mierne zlepšiť presnosť, ale výrazne spomaňuje.

\subsection*{Výsledky}

\vskip 10pt

\begin{table}[H]
\centering
\begin{tabular}{ll}
\toprule
\textbf{Angličtina} & \textbf{Přesnost} \\
\midrule
1. stupeň & 0.8910645842567726 \\
2. stupeň & 0.88881069669247 \\
3. stupeň & 0.8602301723689006 \\
4. stupeň & 0.8931381232741104 \\
5. stupeň & 0.8927185098345708 \\
\midrule
Průměr   & 0.885192417285 \\
$\sigma$ & 0.0125734981194 \\
\midrule
Průmer triv. & 0.865374314735\\
$\sigma$ triv. & 0.0143254860843 \\
\bottomrule
\end{tabular}
\hskip 50pt
\begin{tabular}{ll}
\toprule
\textbf {Čeština} & \textbf{Přesnost} \\
\midrule
1. stupeň & 0.7738987193793506 \\
2. stupeň & 0.7888304373239811 \\
3. stupeň & 0.7535799207397622 \\
4. stupeň & 0.7922554742597533 \\
5. stupeň & 0.7880596223212392 \\
\midrule
Průměr   & 0.779324834805 \\
$\sigma$ & 0.0124922171647 \\
\midrule
Průmer triv. & 0.738888267001\\
$\sigma$ triv. & 0.0134321134672 \\
\bottomrule
\end{tabular}
\caption{Přesnost (accuracy) pro Brillův tagger pro oba jazyky.}
\label{eng_big}
\end{table}

\section*{Task 2}

\subsection*{Nesupervizovaný HMM --- Baum-Welch}

Pre túto časť sme sa rozhodli použiť iba bigramové a unigramové štatistiky, pretože ak by sme nechceli použiť použiť nuly ``natvrdo'' zabralo by to príšerne veľa času pre angličtinu, pre češtinu by to bolo jednoducho nerealizovateľné, časovo ani pamäťovo.

\end{document}
